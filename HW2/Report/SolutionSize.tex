%!TEX root = ./HW2.tex


\section{Solution Size}
As the number of cities to traverse increases, the possible solutions increases exponentially.  The number of the solutions is calculated as $N!$ where $N$ is the number of cities in the tour.  	For all the experiments run in this assignment, 10,000 solutions were generated before stopping the search.  A generated solution could have been a duplicate of a solution already produced.  As a result, the number of solutions searched over cannot be determined for sure.  However, as an upper bound, the uniqueness of solutions will be assumed.

Even with only 15 cities, less than 1 millionth of a percent of the space searched is sufficient to find a near optimal solution.  SA can introduce a significant amount of randomness into a solution early on.  In my implementation of EA, the amount of variability is decreasing throughout the search based on the number of iterations.  However, this noise schedule may not be appropriate.  For larger spaces, large noise should remain for longer.  The reduction in noise is causing search to find a local minimum instead of exploring sufficiently.  MCTS does not seem very well suited for this problem.  The most common use of MCTS is in games that have very large state spaces like Go.  An intelligent playout can help direct the search.  In this example, the playout is just randomly selecting the remaining states.  The benefit of MCTS is that it can combined with another search algorithm to help with playout.  If random playout was instead replaced with SA playout, the results would increase.  Having a good playout algorithm is one feature that allowed google to create AlphaGo.

\begin{table}[H]
\centering
\begin{tabular}{|c|c|c|}
\hline
\begin{tabular}[c]{@{}c@{}}Number of\\ Cities\end{tabular} & Solutions & \begin{tabular}[c]{@{}c@{}}Percent\\ Searched\end{tabular} \\ \hline
15                                                         & 1.31e12   & 7.65e-7\%                                                  \\ \hline
25                                                         & 1.55e25   & 6.45e-20\%                                                 \\ \hline
100                                                        & 9.33e157  & 1.07e-152\%                                                \\ \hline
\end{tabular}
\caption{Comparison of Amount of Space Searched}
\label{tab:SolutionSize}
\end{table}